\chapter{Configurations}\label{configurations}

\section{Introduction}\label{introduction-1}

The Roa Logic AHB-Lite APB4 GPIO is a fully configurable General Purpose
Input/Output core. The core parameters and configuration options are
described in this section.

\section{Core Parameters}\label{core-parameters}

\begin{longtable}[]{@{}lccl@{}}
\toprule
Parameter & Type & Default & Description\tabularnewline
\midrule
\endhead
\texttt{PDATA\_SIZE} & Integer & 8 & APB4 Data Bus \& GPIO Size\tabularnewline
\texttt{INPUT\_STAGES} & Integer & 2 & Number of \texttt{GPIO\_I} input synchronization
stages\tabularnewline
\bottomrule
\caption{Core Parameters}
\end{longtable}

\subsection{PDATA\_SIZE}\label{pdata_size}

The \texttt{PDATA\_SIZE} parameter specifies the width of the APB4 data bus and
corresponding GPIO interface width. This parameter must equal an integer
multiple of bytes.

\subsection{INPUT\_STAGES}\label{input_stages}

The APB4 GPIO inputs are sampled on the rising edge of the APB4 bus clock (\texttt{PCLK}). As these inputs may be asynchronous to the bus clock, the core automatically synchronises these signals and the \texttt{INPUT\_STAGES} parameter determines the number of synchronisation stages. Increasing this parameter reduces the possibility of metastability due to input signals changing state while being sampled, but at the cost of increased latency. The default value of the \texttt{INPUT\_STAGES} parameter is 2

\section{Control Registers}\label{registers}

The APB4 GPIO core implements user accessible registers as described below:

\begin{longtable}[]{@{}lccl@{}}
\toprule
\textbf{Register} & \textbf{Address} & \textbf{Access} & \textbf{Function}\tabularnewline
\midrule
\endhead
\texttt{MODE} & \texttt{Base + 0x0} & Read/Write & Push-Pull or Open-Drain Mode\tabularnewline
\texttt{DIRECTION} & \texttt{Base + 0x1} & Read/Write & Output Enable control\tabularnewline
\texttt{OUTPUT} & \texttt{Base + 0x2} & Read/Write & Output Data Store\tabularnewline
\texttt{INPUT} & \texttt{Base + 0x3} & Read Only & Input Data Store\tabularnewline
\texttt{TRIGGER\_TYPE} & \texttt{Base + 0x4} & Read/Write & Trigger Type\tabularnewline
\texttt{TRIGGER\_LVL0} & \texttt{Base + 0x5} & Read/Write & Trigger Sense 0\tabularnewline
\texttt{TRIGGER\_LVL1} & \texttt{Base + 0x6} & Read/Write & Trigger Sense 1\tabularnewline
\texttt{TRIGGER\_STATUS} & \texttt{Base + 0x7} & Read/Write & Trigger Status\tabularnewline
\texttt{IRQ\_ENABLE} & \texttt{Base + 0x8} & Read/Write & Enable Interrupts\tabularnewline
\bottomrule
\caption{User Registers}
\end{longtable}

\subsection{MODE}\label{mode}

MODE is a PDATA\_SIZE bits wide Read/Write register accessible at the address 0x0. Each bit of the register individually sets the operating mode for each signal of the \texttt{GPIO\_O} and \texttt{GPIO\_OE} buses as either push-pull or open drain, as follows:

\begin{longtable}[]{@{}cl@{}}
\toprule
\textbf{MODE[n]} & \textbf{Operating Mode}\tabularnewline
\midrule
\endhead
0 & Push-Pull\tabularnewline
1 & Open Drain\tabularnewline
\bottomrule
\caption{MODE Register}
\end{longtable}

In push-pull mode, data written to the OUTPUT register directly drives
the output bus \texttt{GPIO\_O}. The \texttt{DIRECTION} register is then used to enable
\texttt{GPIO\_O} to drive the IO pad when set to `Output' mode (`1').

In open-drain mode, \texttt{GPIO\_O} is connected such that the IO Pad is driven low ('0') when the output is '0' and is High-Z, pulled high via an external resistor, when the output is '1'.

\subsection{DIRECTION}\label{direction}

DIRECTION is a \texttt{PDATA\_SIZE} bits wide active-high read/write register, accessible at the address \texttt{0x1}, and controls the output enable bus \texttt{GPIO\_OE[n]}, effectively controlling if \texttt{PAD[n]} operates as an input or an output.

\begin{longtable}[]{@{}cl@{}}
\toprule
\textbf{DIRECTION[n]} & \textbf{Direction}\tabularnewline
\midrule
\endhead
0 & Input\tabularnewline
1 & Output\tabularnewline
\bottomrule
\caption{DIRECTION Register}
\end{longtable}

\subsection{OUTPUT}\label{output}

OUTPUT is a \texttt{PDATA\_SIZE} bits wide read/write register accessible at the address \texttt{0x2}.

Each bit of the OUTPUT register specifies the \texttt{PAD[n]} level when \texttt{GPIO[n]} is programmed as an output. Writing a ‘0’ drives a low level onto \texttt{PAD[n]}, whereas writing a ‘1’ drives a ‘1’ (push-pull) or high-z(open-drain) onto \texttt{PAD[n]}.

\subsection{INPUT}\label{input}

INPUT is a \texttt{PDATA\_SIZE} bits wide read-only register accessible at the address \texttt{0x3}.

On the rising edge of the APB4 Bus Clock (\texttt{PCLK}) input data on pins
\texttt{GPIO\_I} is sampled, synchronised and stored in the \texttt{INPUT} register where it may be read via the APB4 Bus Interface.

\subsection{TRIGGER\_TYPE}\label{triggertype}

TRIGGER\_TYPE is a \texttt{PDATA\_SIZE} bits wide Read/Write register accessible at address \texttt{0x4}.

Each bit of the register sets if a \texttt{GPIO\_I} input bit is configured as a level or edge sensitive interupt trigger as defined below:

\begin{longtable}[]{@{}cl@{}}
\toprule
\textbf{TRIGGER\_TYPE[n]} & \textbf{Type}\tabularnewline
\midrule
\endhead
0 & Level\tabularnewline
1 & Edge\tabularnewline
\bottomrule
\caption{TRIGGER\_TYPE Register}
\end{longtable}

\subsection{TRIGGER\_LVL0 and TRIGGER\_LVL1}\label{triggermode}

TRIGGER\_LVL0 and TRIGGER\_LVL1 are \texttt{PDATA\_SIZE} bits wide Read/Write registers accessible at addresses \texttt{0x5} and \texttt{0x6} respectively.

Each bit of the TRIGGER\_LVL0 and TRIGGER\_LVL1 registers set the trigger sense of an interrupt input on \texttt{GPIO\_I}.  Based on the corresponding TRIGGER\_TYPE bit defining a \texttt{GPIO\_I[n]} input as level or edge triggered, an interrupt is triggered when the \texttt{GPIO\_I[n]} input is low and/or high, or on a rising and/or falling edge transition as documented in the tables below.

Setting both registers to all '0' means that triggers are disabled and no interrupt will be generated from any \texttt{GPIO\_I} input. This is the default state.

\begin{longtable}[]{@{}ccc@{}}
\toprule
\textbf{TRIGGER\_LVL0[n]} & \textbf{Level Triggered} & \textbf{Edge Triggered}\tabularnewline
\midrule
\endhead
0 & no trigger when low & no trigger on falling edge\tabularnewline
1 & trigger when low & trigger on falling edge\tabularnewline
\bottomrule
\caption{TRIGGER\_LVL0 Register}
\end{longtable}

\begin{longtable}[]{@{}ccc@{}}
\toprule
\textbf{TRIGGER\_LVL1[n]} & \textbf{Level Triggered} & \textbf{Edge Triggered}\tabularnewline
\midrule
\endhead
0 & no trigger when high & no trigger on rising edge\tabularnewline
1 & trigger when high & trigger on rising edge\tabularnewline
\bottomrule
\caption{TRIGGER\_LVL1 Register}
\end{longtable}

\subsection{TRIGGER\_STATUS}\label{triggerstatus}

TRIGGER\_STATUS is a \texttt{PDATA\_SIZE} bits wide Read/Write register accessible at address \texttt{0x7}.

Each bit of TRIGGER\_STATUS register is set ('1') if an trigger condition is detected on the corresponding \texttt{GPIO\_I[n]} input according to the settings of TRIGGER\_TYPE and TRIGGER\_LVL0/1. If both TRIGGER\_STATUS[n] and IRQ\_ENABLE[n] are set ('1'), an interrupt is generated on the \texttt{IRQ\_O} pin.

TRIGGER\_STATUS may be read to determine if a trigger condition has occured on the corresponding input. Writing a '1' to TRIGGER\_STATUS[n] will clear the status, unless a new trigger is detect simultaneously, in which case TRIGGER\_STATUS[n] will remain set.

\begin{longtable}[]{@{}cl@{}}
\toprule
\textbf{TRIGGER\_STATUS[n]} & \textbf{Status}\tabularnewline
\midrule
\endhead
0 & no trigger detected/irq pending\tabularnewline
1 & trigger detected/irq pending\tabularnewline
\bottomrule
\caption{TRIGGER\_STATUS Register}
\end{longtable}

\subsection{IRQ\_ENABLE}\label{irqenable}

IRQ\_ENABLE is a \texttt{PDATA\_SIZE} bits wide Read/Write register accessible at address \texttt{0x8}.

Each bit of IRQ\_ENABLE determines if the \texttt{IRQ\_O} pin will be asserted when a trigger condition occurs on the corresponding \texttt{GPIO\_I[n]} input, and so enabling if interupts are generated from the APB4 GPIO core.

\begin{longtable}[]{@{}cl@{}}
\toprule
\textbf{IRQ\_ENABLE[n]} & \textbf{Definition}\tabularnewline
\midrule
\endhead
0 & disable irq generation\tabularnewline
1 & enable irq generation\tabularnewline
\bottomrule
\caption{IRQ\_ENABLE Register}
\end{longtable}
